\chapter{The Simulation Algorithm} % Main chapter title
\label{Chapter5} % Change X to a consecutive number; for referencing this chapter elsewhere, use \ref{ChapterX}


In this chapter the implementation algorithm is presented. The algorithm is
based on the gQMCFFD algorithm from \cite{qmcXray2023} to simulate X-ray images
efficiently, including scattering effects. It is specifically adapted to the
needs of this thesis to analyse and compare the performance of the different
sampling methods.

\section{Forced Fixed Detection}

\ac{ffd} is a method used to simulate X-ray images efficiently, particularly in the context of X-ray imaging systems. It is designed to handle multiple scatter orders and can be used either with Monte Carlo sequences or with Quasi-Monte Carlo (QMC) sequences for sampling the random variables involved in the simulation
%-------------------------------------------------------------------------------
%	SECTION 1
\section{Algorithm overview}
%-------------------------------------------------------------------------------































\pagebreak
.


In this section the gQMCFFD algorithm from \cite{qmcXray2023} is introduced. The gQMCFFD algorithm uses QMC methods to simulate an X-ray image efficiently including scattering effects. The algorithm is designed to handle multiple scatter orders and utilizes 
In this section the Algorithm~\ref{alg:gQMCFFD} from \cite{qmcXray2023} is presented. The \ac{qmc}-Method is used to simulate the X-ray image. Therefore the algorithm is used 
The following pseudoalgorithm outlines the process of simulating X-ray photon transport using Quasi-Monte Carlo (QMC) methods. The algorithm generates a sequence of QMC samples to determine the initial positions and directions of photons, simulates their transport through a defined geometry, and records the results of interactions with materials. By that many other algorithms are used such als the \ac{rita} algorithm.
\begin{algorithm}{gQMCFFD}
\caption{gQMCFFD: X-ray Scatter Simulation (Part 1)}
\label{alg:gQMCFFD}
\begin{algorithmic}[1]
\State \textbf{Input:} Max. scatter order $N$, phantom geometry $\mathcal{P}$, energy spectrum $\phi$, beam angle $\alpha$, set of detector pixels $\mathcal{G}=\{G_1, ..., G_s\}$, QMC point $u^j \in [0,1]^{4N}$, step size $\Delta s$
\vspace{.25cm}

\State \textbf{Initialize photon using } $u^j_1, u^j_2, u^j_3$:
\begin{itemize}
    \item energy $E_0 \sim \phi(E)$ by inverse transform sampling with $u^j_1$
    \item direction $\vec{\omega}_0$ within cone angle $\alpha$ using $u^j_2, u^j_3$
    \item weight $W_0 = I_0(\vec{\omega}_0) = 1$
    \item escape probability $p_0 = 0$ \textcolor{red}{TODO: ist das richtig initialisiert?}
    \end{itemize}
    \State Compute entry point: find smallest $t_0$ s.t. $A_0 = S + t_0 \cdot \vec{\omega}_0 \in \partial\mathcal{P}$

\vspace{.25cm}
\State Initialize: $f_{n,k} = 0$ for all $D_k \in \mathcal{G}$

\For{$i = 1$ to $N$} \State $t_i\gets\Delta s$ \Comment{Initialize path length}
    \State$A_i \gets A_{i-1} + t_i \cdot \vec{\omega}_{i-1}$ \Comment{Initialize
    position} \LineComment{
    \parbox[t]{\dimexpr\linewidth-\algorithmicindent}{Sample free path length
    $t_i$ with $u^j_{4i}$}}
    
    \While{$A_{i} \in \mathcal{P}$ \textbf{and} $\int\limits_0^{t_i}
    \mu_{\text{tot}}(A_{i-1} + s\vec{\omega}_{i-1}, E_{i-1}) ds < -\ln{\big(1 -
    (1 - p_{i-1}) u^j_{4i} \big)}$} \State $t_i \gets t_i + \Delta s$
    \Comment{Update path length} \State $A_i \gets A_{i-1} + t_i \cdot
    \vec{\omega}_{i-1}$ \Comment{Update position} \EndWhile
    
    \LineComment{Sample interaction type}
    \If{$\mu_{\text{tot}}(A_i, E_i) \cdot u^j_{4i+3} < \mu_{\text{comp}}(A_i, E_i)$}

        \State $\delta^i=0$ $\to$ \textbf{Compton Scattering}
        \State $p_{y_0}$ gemäß Gleichung (5)
        \State $\vec{\omega}_{i} \gets \text{comptonDirectionSampling}(E_{i-1}, \vec{\omega}_{i-1}, u^j_{4i+1}, u^j_{4i+2})$

        \ElsIf{$\mu_{\text{comp}}(A_i, E_i) \leq \mu_{\text{tot}}(A_i, E_i)
        \cdot u^j_{4i+3} < \mu_{\text{comp}}(A_i, E_i) + \mu_{\text{ray}}(A_i,
        E_i)$}
        
        \State $\delta^i=1$ $\to$ \textbf{Rayleigh Scattering} \State
        $p_{y_1}$ gemäß Gleichung (6)

        \Else 
            \State \textbf{Photoelectric Absorption} $\to$ \textbf{break}
    \EndIf
    \State Sample new direction $\vec{\omega}_i$ using RITA using randoms $u^j_{4i+1}, u^j_{4i+2}$
    \State \textcolor{red}{TODO: herausfinden, wie neue Energie berechnet wird}
    \State Compute escape probability along $\vec{\omega}_{i-1}$:
    $$p_{i-1} = \exp\left(-\int_0^{c_{i-1}} \mu_{\text{tot}}(A_{i-1} + s\vec{\omega}_{i-1}, E_{i-1}) ds\right)$$
    
    \State Update weight: $W_i = W_{i-1} \cdot (1 - p_{i-1})$

    \For{each detector pixel $D_j \in \mathcal{G}$}
        \State Determine forced direction $\vec{\omega}_{i,j}$ from $A_i \to D_j$
        \State Compute transmission factor:
        \[
        T_{i,j} = \exp\left(-\int_0^{b_{i,j}} \mu_{\text{tot}}(A_i + s\vec{\omega}_{i,j}, E_i) ds\right)
        \]
        \State Compute directional scatter PDF: $p^y(A_i, E_{i-1} \rightarrow E_i, \vec{\omega}_{i-1} \rightarrow \vec{\omega}_{i,j})$
        \State Update scatter contribution:
        \[
        f_{n,j} \mathrel{+}= W_i \cdot p^y \cdot T_{i,j}
        \]
    \EndFor

\EndFor

\algstore{myalg}
\end{algorithmic}
\end{algorithm}

\begin{algorithm}
\caption{gQMCFFD: X-ray Scatter Simulation (Part 2)}
\begin{algorithmic}[1]
\algrestore{myalg}

\vspace{.25cm}
\State \textbf{Primary intensity (if unscattered):}
\For{each detector pixel $D_j \in \mathcal{G}$}
    \State Determine direct line $\vec{\omega}_{0,j}$ from $A_0$ to $D_j$
    \State Compute:
    \[
    T_{0,j} = \exp\left(-\int_0^{b_{0,j}} \mu_{\text{tot}}(A_0 + s\vec{\omega}_{0,j}, E_0) ds\right)
    \]
    \State Add primary contribution:
    \[
    f_{n,j} \mathrel{+}= W_0 \cdot T_{0,j}
    \]
\EndFor

\State \textbf{Output:} $f_{n,j}$ for each detector pixel $D_j \in \mathcal{G}$

\end{algorithmic}
\end{algorithm}