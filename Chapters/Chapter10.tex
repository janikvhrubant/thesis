%!TEX root = ../main.tex
\chapter{The Simulation Algorithm}
\label{Chapter5}

This chapter presents the algorithm used for X-ray image simulation. It
incorporates elements of \emph{Forced Detection} as introduced in \cite{fd2001},
in order to accelerate convergence.

Unlike standard Monte Carlo simulations, where a random variable determines
whether a photon escapes the phantom or undergoes another scattering event, the
forced detection approach employed, calculates the probability of escaping the
phantom at each interaction point and reflects this distribution in accordingly
updated photon intensities for the case of escaping the phantom and the case of
scattering. Instead of probabilistically terminating the photon trajectory, the
algorithm tracks it through a predefined number of scattering events $N$. At
each interaction, the remaining intensity is updated to reflect both the
probability of Compton scattering and the conditional escape probability. This
ensures that both potential outcomes—escape or continued scattering—are
implicitly accounted for in the evolving photon intensity. As a result, the
simulation maintains physical accuracy while achieving a significant reduction
in variance.

When a photon eventually escapes the phantom, it contributes to the
corresponding detector pixel, weighted by its current intensity, photon energy,
and escape probability. Conversely, the distance to the next interaction is
sampled based on the probability of not escaping the phantom. Further
implementation details are discussed in Section~\ref{sec:algorithmOverview}.

This variance reduction strategy enables the simulation to converge more rapidly
toward high-resolution images with suppressed noise and well-preserved
structural detail.


%-------------------------------------------------------------------------------
%	SECTION 1
\section{Algorithm overview}
\label{sec:algorithmOverview}
%-------------------------------------------------------------------------------

The algorithm begins by initializing a set of photons, each assigned an initial
energy $E_0$, sampled from the X-ray tube spectrum and an initial direction
$\vec{\omega}_0$ sampled uniformly within a cone centered around the principal
beam axis. A total of three random variables are drawn for each photon to
determine its energy and direction.

Given a fixed source position $A$ representing the location of the X-ray tube,
each photon's initial origin is placed within a voxel grid composed of cubic
voxels that define the geometry of the scene. Each voxel encodes an integer
value identifying a specific material or tissue type. Furthermore, each photon
is initialized with an intensity of $W_0=1$, which is iteratively updated
throughout the simulation to account for attenuation and scattering processes.

\begin{figure}[H]
    \centering
    \tdplotsetmaincoords{70}{120} % view angle
    \begin{tikzpicture}[tdplot_main_coords, scale=.7]

    % Grid size
    \def\N{3}

    % Draw voxels
    \foreach \x in {0,...,2} {
        \foreach \y in {0,...,2} {
            \foreach \z in {0,...,2} {
                \draw[gray, thick] (\x,\y,\z) -- ++(1,0,0) -- ++(0,1,0) -- ++(-1,0,0) -- cycle; % bottom face
                \draw[gray, thick] (\x,\y,\z) -- ++(0,0,1); % vertical lines from bottom
                \draw[gray, thick] (\x+1,\y,\z) -- ++(0,0,1);
                \draw[gray, thick] (\x,\y+1,\z) -- ++(0,0,1);
                \draw[gray, thick] (\x+1,\y+1,\z) -- ++(0,0,1);
                \draw[gray, thick] (\x,\y,\z+1) -- ++(1,0,0) -- ++(0,1,0) -- ++(-1,0,0) -- cycle; % top face
            }
        }
    }

    % Optional axis labels
    \draw[->, thick] (0,0,0) -- (3.5,0,0) node[anchor=north east]{$x$};
    \draw[->, thick] (0,0,0) -- (0,3.5,0) node[anchor=north west]{$y$};
    \draw[->, thick] (0,0,0) -- (0,0,3.5) node[anchor=south]{$z$};

    \end{tikzpicture}
\end{figure}

Then an algorithm is applied to traverse the photon through the voxel grid until
it reaches the first material which is not air. This is done with a ray
traversal algorithm. Once the phantom at a point $A_0$ is reached, the photon
transport is simulated by the physical laws described in Chapter~\ref{Chapter6}.

Hereby, as in all other scatter points within the phantom, the free \emph{free
path length} $t_i$ is sampled from the exponential distribution based on
\emph{Beer-Lambert's law} (Eq.~\ref{eq:lambert_beer_law}) and the \emph{total
attenuation coefficient} $\mu(A_i, E_i)$ accordingly. $A_i$ hereby denotes the
current position and $E_i$ the current energy. First, the \emph{escape
probability} depending on the current position $A_i$ and the unit direction
$\vec{\omega}_i$ and the energy $E_i$ is computed. The escape probability in
Equation~\ref{eq:escapeProbability} is the probability of the photon escaping
the phantom at the current position $A_i$ without being further scattered.

\begin{equation}
    \label{eq:escapeProbability}
    p(A_i, \omega_i, E_i) = \exp\bigg(-\int\limits_{\overrightarrow{A_iC_i}} \mu(x, E_i) dx\bigg)
\end{equation}

Hereby $C_i$ denotes the exit point of the phantom in the direction of the photon $\vec{\omega}_i$. The escape probability is used to determine the intensity of the photon without the $(i+1)$-th scatter event.

Using one random variable $u_{3i+1}$, the free path length is sampled. Hereby both cases are being considered:

\begin{enumerate}
    \item \textbf{Photon escapes the phantom without further scattering:}\\
        This happens with the portion matching the escape probability $p(A_i,
        \omega_i, E_i)$. Accordingly 
        \begin{equation}
            W^{\text{escape}}_{i+1} = W_i \cdot p(A_i, \omega_i, E_i)
        \end{equation}
        The photon contributes to the detector signal in direction
        $\vec{\omega}_i$ with a final intensity of $E_i \cdot
        W^{\text{escape}}_{i+1}$. In case $i=0$, this intensity is accounted as
        priamry intensity.

    \item \textbf{Photon does not escape the phantom and scatters:}\\
        This happens with the portion matching the inverse of the escape
        probability $1 - p(A_i, \omega_i, E_i)$. Accordingly, the free path
        length is sampled from the exponential distribution by solving
        Equation~\ref{eq:freePathLength} for $t_i$:
        \begin{equation}
            \label{eq:freePathLength}
            \exp\bigg(-\int\limits_0^{t_i} \mu(A_{i} + s \cdot \vec{\omega}_
            {i}, E_{i}) ds\bigg) = u_{3i+4}
        \end{equation}
        Accordingly, the next scatter point is being computed as:
        \begin{equation}
            A_{i+1} = A_i + t_i \cdot \vec{\omega}_i
        \end{equation}
        The photon intensity is then updated with:
        \begin{equation}
            W_{i+1} = W_i \cdot (1 - p(A_i, \omega_i, E_i)) \cdot \frac{\mu_
            {\text{CS}}(A_{i+1}, E_i)}{\mu(A_{i+1}, E_i)}
        \end{equation}
\end{enumerate}


In the next step, the photon energy and opening angle after the Compton scatter
event is sampled with a second random variable $u_{3i+2}$. A thrid variable is
used to apply a azimuthal rotation around the direction of the photon
$\vec{\omega}_i$ to sample the new direction $\vec{\omega}_{i+1}$ of the photon
after the scatter event.

This process is repeated according to the maximum scatter order $N$.

\textcolor{red}{TODO: what happens with the leftover intensity? It is forwarded without any further scattering!}

%-------------------------------------------------------------------------------
%	SECTION 2
\section{Sub-Algorithms}
\label{sec:subAlgorithms}
%-------------------------------------------------------------------------------

To model the relevant physical processes in a structured manner, the main
simulation is decomposed into several sub-algorithms. This section describes the
purpose and implementation of each sub-algorithm in detail.

\subsection{Photon Generation}
For the photon generation step, two fundamental properties are sampled for each
photon:

\begin{itemize}
    \item \emph{Photon energy}, sampled using a single random variable from the
    X-ray spectrum.
    \item \emph{Photon direction}, sampled using two random variables uniformly
    within a cone defined by the beam axis $\vec{d}$ and opening angle $\alpha$.
\end{itemize}

\subsubsection{Photon Energy Sampling}

The photon energy is sampled from the X-ray tube spectrum, which is represented
as a discrete set of energy values with corresponding fluence values. The
spectrum was generated using \emph{Spekpy}~\cite{spekpy,poludniowski2021spekpy},
based on an X-ray tube configured with the parameters listed in
Table~\ref{tab:xray_params}.

\begin{table}[H]
    \centering
    \begin{tabular}{ll}
        \toprule
        \textbf{Parameter}      & \textbf{Value} \\
        \midrule
        Tube voltage            & 120\,kV \\
        Anode material          & Tungsten \\
        Filtration              & 0.4\,\si{\milli\meter} Tin (Sn),
        0.1\,\si{\milli\meter} Copper (Cu) \\
        Target angle            & 12.5\textdegree \\
        \bottomrule
    \end{tabular}
    \caption{Parameters used to generate the X-ray spectrum with \emph{Spekpy}.}
    \label{tab:xray_params}
\end{table}

Photon energies are sampled using inverse transform sampling. The algorithm
normalizes the fluence values to obtain a probability density function (PDF),
computes the corresponding cumulative distribution function (CDF) and uses a
uniformly distributed random variable to select an energy according to the CDF.

\begin{algorithm}[H]
\caption{Photon Energy Sampling from Spectrum}
\label{alg:photonEnergySampling}
\begin{algorithmic}[1]
\Require Array of energies $E = [E_1, E_2, \dots, E_n]$ 
\Require Corresponding fluence values $\Phi = [\phi_1, \phi_2, \dots, \phi_n]$ 
\Require Number of samples $N$ 
\Require Random variable $u \sim \mathcal{U}(0, 1)$ \Ensure Sampled photon
energies $S = [s_1, s_2, \dots, s_N]$

\LineComment{Normalize fluence values:} \State $$T \gets \sum_{i=1}^{n} \phi_i$$
\State $$\text{PDF}[i] \gets \frac{\phi_i}{T}$$

\LineComment{Compute cumulative distribution function:} 
\State $$\text{CDF}[1] \gets \text{PDF}[1]$$ 
\For{$i = 2$ to $n$} 
\State $\text{CDF}[i] \gets \text{CDF}[i-1] + \text{PDF}[i]$ 
\EndFor

\LineComment{Note: $PDF[i]>0$ in the spectrum, therefore $\text{CDF}$ is
strictly increasing.}

\State Create interpolating function $\text{ InverseCDF}(u)$ from
$(\text{CDF}[i], E[i])$

\State \Return $\text{InverseCDF}(u)$

\end{algorithmic}
\end{algorithm}

\subsubsection{Photon Direction Sampling}

The direction of each photon is sampled uniformly within a cone defined by the
beam axis $\vec{d}$ and opening angle $\alpha$. This requires two independent
random variables $u_1, u_2 \sim \mathcal{U}(0, 1)$.

Algorithm~\ref{alg:uniformDirectionSampling}, adapted
from~\cite{venkatapathi2021n}, generates a random unit vector $\vec{v}$
uniformly distributed within a cone of half-angle $\alpha$ around the direction
$\vec{d}$:

\begin{enumerate}
    \item \textbf{Sampling the polar angle $\theta$}:\\
    Draw $u_1 \sim \mathcal{U}(0,1)$ and compute
    \[
        \theta = \arccos\left(1 - u_1 (1 - \cos \alpha)\right).
    \]
    This corresponds to inverse transform sampling such that $\cos\theta$ is
    uniformly distributed on $[\cos\alpha, 1]$, resulting in uniform sampling
    over the spherical cap.

    \item \textbf{Sampling the azimuthal angle $\phi$}:\\
    Draw $u_2 \sim \mathcal{U}(0,1)$ and compute
    \[
        \phi = 2\pi u_2,
    \]
    ensuring a uniform distribution around the cone axis.

    \item \textbf{Constructing the local direction vector}:\\
    In a local spherical coordinate system with the cone axis aligned along the
    $z$-axis, the sampled unit vector is
    \[
        \vec{v}_{\text{local}} =
        \begin{pmatrix}
            \sin\theta \cos\phi \\
            \sin\theta \sin\phi \\
            \cos\theta
        \end{pmatrix}.
    \]

    \item \textbf{Rotation into global coordinates}:\\
    To align the cone axis from the local $z$-axis to an arbitrary unit vector
    $\vec{d}$, an orthonormal basis $(\vec{u}, \vec{v}, \vec{d})$ is constructed
    via the Gram–Schmidt process:
    \begin{itemize}
        \item Choose a helper vector $\vec{a} = (0,0,1)$ if $|d_3| < 0.999$,
        otherwise $\vec{a} = (1,0,0)$.
        \item Compute $\vec{u} = \frac{\vec{a} \times \vec{d}}{\|\vec{a} \times
        \vec{d}\|}$.
        \item Set $\vec{v} = \vec{d} \times \vec{u}$ to complete a right-handed
        orthonormal basis.
    \end{itemize}
    Finally, rotate $\vec{v}_{\text{local}}$ into global coordinates via the
    transformation
    \[
        \vec{v}_{\text{global}} = (\vec{v}_{\text{local}})_x \vec{u} + (\vec{v}_{\text{local}})_y \vec{v} + (\vec{v}_{\text{local}})_z \vec{d}.
    \]
\end{enumerate}


\begin{algorithm}[H]
\caption{Uniform Direction Sampling Within a Cone}
\label{alg:uniformDirectionSampling}
\begin{algorithmic}[1]
\Require Cone angle $\alpha$
\Require Unit beam direction vector $\vec{d} = (d_1, d_2, d_3)$
\Require Random variables $u_1, u_2 \sim \mathcal{U}(0,1)$
\Ensure Sampled direction vector $\vec{v}$ uniformly within cone around
$\vec{d}$

\LineComment{Calculate angles according to samples}

\State $\theta \gets \arccos(1 - u_1 (1 - \cos \alpha))$ \Comment{Polar angle}
\State $\phi \gets 2\pi u_2$ \Comment{Azimuthal angle}

\LineComment{Calculate local direction vector in spherical coordinates}

\State $$\vec{v}_{\text{local}} \gets 
\begin{pmatrix}
\sin\theta \cos\phi \\
\sin\theta \sin\phi \\
\cos\theta \end{pmatrix}$$

\LineComment{Orthonormal basis construction (Gram-Schmidt)}
\If{$|d_3| < 0.999$}
    \State $\vec{a} \gets (0, 0, 1)$
\Else
    \State $\vec{a} \gets (1, 0, 0)$
\EndIf
\State $\vec{u} \gets \frac{\vec{a} \times \vec{d}}{\|\vec{a} \times \vec{d}\|}$ \Comment{Orthogonal vector}
\State $\vec{v} \gets \vec{d} \times \vec{u}$ \Comment{Complete right-handed basis}

\LineComment{Rotate local vector into global coordinates}
\State $\vec{v}_{\text{global}} \gets \vec{u} (\vec{v}_{\text{local}})_x + \vec{v} (\vec{v}_{\text{local}})_y + \vec{d} (\vec{v}_{\text{local}})_z$

\State \Return $\vec{v}_{\text{global}}$

\end{algorithmic}
\end{algorithm}


\subsection{Ray Traversal}
\label{sec:rayTraversal}

The ray traversal algorithm is responsible for simulating the propagation of a
photon through the voxel grid. It is used to forward the photon until it reaches
the first chemical compound, which is not air (or is air). It is also
used to simulate the photon transport within the phantom until it reaches the
exit point of the phantom.

The Input values of the algorithm are:
\begin{itemize}
    \item The voxel grid $materialGrid$ representing the geometry of the scene.
    \item The initial position of the photon $A$.
    \item The (unit) direction of the photon $\vec{\omega}$.
\end{itemize}

The algorithm traverses the photon through the voxel grid, voxel by voxel while checking the material of the next voxel and iterrupts in case the next voxel is not air (or is air).

The return values of the algorithm are:
\begin{itemize}
    \item The coordinates of the final postion: $C$
    \item An array of all crossed voxel indices: $crossedVoxels$
    \item An array of all crossed voxel materials: $crossedMaterials$
    \item An array of al entry points of each crossed voxel: $entryPoints$
    \item An array of the distances traveres in each voxel: $distances$
    \item A boolean mask indicating whether the photon exited the grid:
    $exitGrid$
\end{itemize}

The sequence of the algorithm can be summarized as follows:
\begin{enumerate}[label=\Roman*.]
    \item Initialize empty arrays for $crossedVoxels$, $crossedMaterials$, $entryPoints$ and $distances$.
    \item Convert the photon position $A$ into voxel coordinates $pos$.
    \item Determine the indices of the next voxel $voxelIdx$ based on the
    $pos$ and the direction $\vec{\omega}$.
    \item Set $exitGrid$ to false.
    \item \textbf{If} the $voxelIdx$ of the next voxel is out of bounds, set $exitGrid$ to true and exit the algorithm.
    \item Determine $material$ of the next $voxelIdx$
    \item \textbf{If} the $material$ is not air, exit the algorithm.
    \item \textbf{While} $material$ is air:
        \begin{enumerate}[label=\arabic*.]
            \item Append $voxelIdx$ to $crossedVoxels$.
            \item Append $material$ to $crossedMaterials$.
            \item Append $pos$ to $entryPoints$.
            \item Determine $maxDist$ to the next voxel boundary.
            \item Append $maxDist$ to $distances$.
            \item Update $pos$ by adding $\vec{\omega} \cdot maxDist$.
            \item Determine $voxelIdx$ of the next voxel based on the updated $pos$.
            \item \textbf{If} the $voxelIdx$ is out of bounds, set $exitGrid$ to true and exit the algorithm.
            \item Determine $material$ of the next $voxelIdx$.
        \end{enumerate}
\end{enumerate}

When the algorithm finishes, the final Position $C$ is set to the last value of
$position$.

To use this agorithm for travering the photons through air until they reach the phantom and to traverse the photons through the phantom until they reach the exit point of the phantom, the algorithm is called with the boolean value $throughAir$. In case $throughAir=true$, the algorithm will traverse the photons through air until they reach the first material which is not air. In case $throughAir=false$, the algorithm will traverse the photons through the phantom until they reach the exit point of the phantom.

Therefore we define a short helper function in
Algorithm~\ref{alg:rayTraversalHelper} beforehand, which generates the specific
expression depending on the boolean value of $throughAir$.

\begin{algorithm}[H]
\caption{Ray Traversal Helper Function}
\label{alg:rayTraversalHelper}
\begin{algorithmic}[1]
\Require Boolean $throughAir$
\Require $material$
\Ensure Boolean value
\If {$throughAir$}
    \State \Return $material = \text{air}$
\Else
    \State \Return $material \neq \text{air}$
\EndIf
\end{algorithmic}
\end{algorithm}

Algorithm~\ref{alg:rayTraversal} implements the process of ray traversing
through the voxel grid in detail.

\begin{algorithm}[H]
    \caption{Ray Traversal Algorithm}
    \label{alg:rayTraversal}
    \begin{algorithmic}[1]
        \Require Boolean $throughAir$
        \Require Material grid $materialGrid \in \mathbb{Z}^{X \times Y \times
        Z}$, voxel size $s \in \mathbb{R}^+$
        \Require Initial pos $A \in \mathbb{R}^3$, direction $\vec{\omega} \in
        \mathbb{R}^3$
        \Ensure Final pos $C \in \mathbb{R}^3$
        \Ensure Arrays $crossedVoxels, crossedMaterials, entryPoints, distances$
        \Ensure Boolean $exitGrid$ indicating whether the photon exited the grid
        \State $pos \gets O / s$
        \State $exitGrid \gets \text{false}$
        \State $currentVoxelIdx \gets \lfloor pos \rfloor$
        \State $negDir$ $\gets \vec{\omega} < 0$
        \State $onB \gets (pos = \lfloor currentVoxelIdx \rfloor)$
        \State $nextVoxelIdx \gets currentVoxelIdx$
        \State $nextVoxelIdx[negDir \land onB] \gets nextVoxelIdx[negDir
        \land onB] - 1$
        \If{any($nextVoxelIdx < 0$) or any($nextVoxelIdx \geq
        \text{shape}(materialGrid)$)}
            \State $exitGrid \gets \text{true}$
            \State $C \gets pos$
            \State \Return $C, crossedVoxels, crossedMaterials,entryPoints, 
            distances, exitGrid$
        \EndIf
        \State $material \gets materialGrid[nextVoxelIdx]$
        \While{Algorithm~\ref{alg:rayTraversalHelper}(throughAir, material)}
            \State $crossedVoxels.\text{append}(nextVoxelIdx)$
            \State $crossedMaterials.\text{append}(material)$
            \State $entryPoints.\text{append}(pos)$
            \State $fracPos \gets pos - \lfloor pos \rfloor$
            \State $maxDist \gets [\infty, \infty, \infty]$
            \State $negDir$ $\gets \vec{\omega} < 0$
            \State $posDir$ $\gets \vec{\omega} > 0$
            \State $onB \gets (pos = \lfloor currentVoxelIdx \rfloor)$
            \State $maxDist[negDir \land onB] \gets -1 / \vec{\omega}[negDir
            \land onB]$
            \State $maxDist[negDir \land \neg onB] \gets 
            -fracPos[negDir \land \neg onB] / \vec{\omega}
            [negDir \land \neg onB]$
            \State $maxDist[posDir] \gets (1 - fracPos[posDir]) / \vec{\omega}
            [posDir]$
            \State $distance = \min(maxDist)$
            \State $distances.\text{append}(maxDist)$
            \State $pos \gets pos + \vec{\omega} \cdot distance$
            \State $currentVoxelIdx \gets \lfloor pos \rfloor$
            \State $nextVoxelIdx \gets currentVoxelIdx$
            \State $nextVoxelIdx[negDir \land onB] \gets nextVoxelIdx[negDir
            \land onB] - 1$
            \If{any($nextVoxelIdx < 0$) or any($nextVoxelIdx \geq
            \text{shape}(materialGrid)$)}
                \State $exitGrid \gets \text{true}$
                \State $C \gets pos$
                \State \Return $C, crossedVoxels, crossedMaterials, entryPoints,
                distances, exitGrid$
            \EndIf
            \State $material \gets materialGrid[nextVoxelIdx]$
        \EndWhile
        \State $C \gets pos$
        \State \Return $C, crossedVoxels, crossedMaterials, entryPoints, distances, exitGrid$

    \end{algorithmic}
\end{algorithm}

\subsection{Forced Detection}

The forced detection algorithm is responsible to apply one iteration of the
forced detection process to a photon within the phantom. Hereby the algorithm
accounts:

\begin{itemize}
    \item The voxel grid $materialGrid$ representing the geometry of the scene.
    \item The voxel grid $totalAttenuationGrid$ representing the attenuation coefficients of the materials in the voxel grid $materialGrid$.
    \item The voxel grid $comptonAttenuationGrid$ representing the Compton scattering coefficients of the materials in the voxel grid $materialGrid$.
    \item The voxel grid $absorptionGrid$ representing the absorption coefficients of the materials in the voxel grid $materialGrid$.
    \item The initial position of the photon $A_i$.
    \item The (unit) direction of the photon $\vec{\omega}_i$.
    \item The initial energy of the photon $E_i$.
    \item The initial intensity of the photon $W_i$.
    \item A random variable $u \sim \mathcal{U}(0, 1)$ to sample the free path length.
    \item The voxel size $s$.
\end{itemize}

The algorithm then applies the ray traversal algorithm
(Algorithm~\ref{alg:rayTraversal}) to traverse the photon through the phantom
until it reaches the exit point of the phantom $C_i$. Obtaining the exit point
$C_i$, the arrays of $crossedVoxels$ and $distances$, now the attenuation
coefficients of the materials along the ray can be extracted from the voxel
grids for the crossed voxels:

\begin{align*}
    \label{eq:gatAttenuation}
    totalAttenuationCoefficients &= totalAttenuationGrid[crossedVoxels] \\
    comptonScatteringCoefficients &= comptonAttenuationGrid[crossedVoxels]
\end{align*}

With the following helper algorithm (Algorithm~\ref{alg:partialProductSum}), the $escapeProbability$ is computed. Further the last distance index $k$ and last interpolation factor $f$ are returned, to solve Equation~\ref{eq:freePathLength} for the free path length $t_i$, which can then be simply computed by:

\begin{equation}
    t_i = \sum_{j=0}^{k-1} distances[j] + f
\end{equation}

Now the next scatter point $A_{i+1}=A_i+t_i\cdot s \cdot\vec{\omega}_i$ can be
determined and the new intensities together with the exit point $C_i$ can be
computed:

\begin{align*}
    voxelIdx &= \lfloor A_{i+1} / s \rfloor \\
    W_{i+1} &= W_i \cdot (1 - escapeProbability) \cdot \frac
    {comptonAttenuationGrid[voxelIdx]}{totalAttenuationCoefficients[voxelIdx]} 
    \\
    W^{\text{escape}}_{i+1} &= W_i \cdot escapeProbability
\end{align*}

And accordingly the forced detection algorithm returns the following values:
\begin{itemize}
    \item The new position of the photon $A_{i+1}$.
    \item The new intensity of the photon after the scatter event $W_{i+1}$.
    \item The intensity of the photon for the case of escaping the phantom $W^{\text{escape}}_{i+1}$.
    \item The exit point of the phantom $C_i$.
    \item The Compton scattering coefficient $\mu_{\text{CS}}$ at the scatter
    point $A_{i+1}$.
\end{itemize}

\begin{algorithm}[!tp]
\caption{Partial Product Sum for Free Path Sampling in Forced Detection}
\label{alg:partialProductSum}
\begin{algorithmic}[1]
\Require Arrays $distances, totalAttenuationCoefficients \in \mathbb{R}^n$, weight $u \in [0, 1]$
\Ensure Last distance index $k$, last interpolation factor $f$,
$escapeProbability$

\State $S_\text{total} \gets 0$
\State Initialize array $P[0\ldots n-1]$

\For{$i \gets 0$ to $n-1$}
    \State $P[i] \gets distances[i] \cdot totalAttenuationCoefficients[i]$
    \State $S_\text{total} \gets S_\text{total} + P[i]$
\EndFor

\State $T \gets u \cdot S_\text{total}$
\State $S \gets 0$

\For{$i \gets 0$ to $n-1$}
    \If{$S + P[i] > T$}
        \State $f \gets \frac{T - S}{totalAttenuationCoefficients[i]}$
        \State \textbf{return} $(i, f, S_\text{total})$
    \EndIf
    \State $S \gets S + P[i]$
\EndFor

\State \textbf{return} $(n, 1.0, S_\text{total})$ \Comment{u = 1.0 or exact fit}
\end{algorithmic}
\end{algorithm}


\begin{algorithm}[H]
\caption{Forced Detection Algorithm}
\label{alg:forcedDetection}
\begin{algorithmic}[1]
\Require Material grid $materialGrid \in \mathbb{Z}^{X \times Y \times Z}$
\Require Total attenuation grid $totalAttenuationGrid \in \mathbb{R}^{X \times 
Y \times Z}$
\Require Compton scattering grid $comptonAttenuationGrid \in \mathbb{R}^{X 
\times Y \times Z}$
\Require Absorption grid $absorptionGrid \in \mathbb{R}^{X \times Y \times Z}$
\Require Initial position $A_i \in \mathbb{R}^3$
\Require Unit direction $\vec{\omega}_i \in \mathbb{R}^3$
\Require Initial energy $E_i \in \mathbb{R}^+$
\Require Initial intensity $W_i \in \mathbb{R}^+$
\Require Random variable $u \sim \mathcal{U}(0, 1)$
\Require Voxel size $s \in \mathbb{R}^+$
\Ensure New position $A_{i+1} \in \mathbb{R}^3$
\Ensure New intensity $W_{i+1} \in \mathbb{R}^+$
\Ensure Escape intensity $W^{\text{escape}}_{i+1} \in \mathbb{R}^+$
\Ensure Exit point $C_i \in \mathbb{R}^3$
\Ensure Compton Attenuation coefficient $\mu_{\text{CS}}$
\State $C_i, crossedVoxels, crossedMaterials, entryPoints, distances, exitGrid
\gets \text{Algorithm~\ref{alg:rayTraversal}}(throughAir = true, materialGrid,
s, A_i)$
\If{exitGrid}
    \State \Return $(C_i, W_i, 0, C_i)$ \Comment{Photon escaped the phantom}
\EndIf
\State $totalAttenuationCoefficients \gets totalAttenuationGrid[crossedVoxels]$
\State $comptonScatteringCoefficients \gets comptonAttenuationGrid
[crossedVoxels]$
\State $absorptionCoefficients \gets absorptionGrid[crossedVoxels]$
\State $k, f, escapeProbability \gets \text{Algorithm~\ref{alg:partialProductSum}}
(distances, totalAttenuationCoefficients, u)$
\State $escapeProbability \gets \frac{escapeProbability}{\sum_{j=0}^{k-1} distances
[j] \cdot totalAttenuationCoefficients[j]}$
\State $t_i \gets \sum_{j=0}^{k-1} distances[j] + f$
\State $A_{i+1} \gets A_i + t_i \cdot s \cdot \vec{\omega}_i$
\State $voxelIdx \gets \lfloor A_{i+1} / s \rfloor$
\State $\mu_{\text{CS}} \gets comptonScatteringCoefficients[voxelIdx]$
\State $W_{i+1} \gets W_i \cdot (1 - escapeProbability) \cdot
\frac{\mu_{\text{CS}}}{totalAttenuationCoefficients [voxelIdx]}$
\State $W^{\text{escape}}_{i+1} \gets W_i \cdot escapeProbability$
\State \Return $(A_{i+1}, W_{i+1}, W^{\text{escape}}_{i+1}, C_i,
\mu_{\text{CS}})$
\end{algorithmic}
\end{algorithm}


\subsection{Photon Exit Point Determination}

To determine the exit point of the photon in the world after exiting the phantom, a more efficient algorithm than the ray traversal algorithm from Section~\ref{sec:rayTraversal} is used can be applied. This algorithm yields the exit point of the phantom which is used to determin the detector pixel the photon contributes to.

The algorithm is initialized with the following parameters:
\begin{itemize}
    \item The voxel grid shape $(N_x,N_y,N_z)$ of the $materialGrid$
    representing the
    geometry of the scene.
    \item The initial position of the photon $C^{\text{phantom}}$.
    \item The (unit) direction of the photon $\vec{\omega}$.
    \item The voxel size $s$.
\end{itemize}

The algorithm then computes the point where the photon exits the voxel grid
efficiently and returns the exit point $C^{\text{grid}}$ and the according
Coordinates $C^{\text{gridCoords}}$. The algorithm is implemented as follows:

\begin{algorithm}[H]
\caption{Compute Exit Point of Ray from Voxel Grid}
\label{alg:exitPointComputation}
\begin{algorithmic}[1]
\Require Ray origin $C^{\text{phantom}} \in \mathbb{R}^3$, direction $\vec{\omega} \in \mathbb{R}^3$
\Require Grid shape $(N_x, N_y, N_z)$, voxel size $s \in \mathbb{R}^+$
\Ensure Exit point $C^\text{grid}$, $C^{\text{gridCoords}}$
\State $C^{\text{phantomCoords}} \gets C^{\text{phantom}} / s$ \Comment{Convert to voxel coordinates}
\State $x_{\min} \gets 0$, $x_{\max} \gets N_x$
\State $y_{\min} \gets 0$, $y_{\max} \gets N_y$
\State $z_{\min} \gets 0$, $z_{\max} \gets N_z$

\For{axis in \{x, y, z\}}
    \State $o \gets C^{\text{phantomCoords}}_\text{axis}$
    \State $d \gets \vec{\omega}_\text{axis}$

    \If{$d > 0$}
        \State $t^{(\text{axis})} \gets \frac{x_{\max} - o}{d}$
    \ElsIf{$d < 0$}
        \State $t^{(\text{axis})} \gets \frac{x_{\min} - o}{d}$
    \ElsIf{$d = 0$}
        \State $t^{(\text{axis})} \gets -\infty$
    \EndIf
\EndFor
\State $t_\text{exit} \gets \min(t^{(x)}, t^{(y)}, t^{(z)})$ \Comment{Exit time}
\State $C^{\text{gridCoords}} \gets C^{\text{phantomCoords}} + t_\text{exit} \cdot \vec{\omega}$
\State $C^{\text{grid}} \gets C^{\text{gridCoords}} \cdot s \cdot \vec{\omega}$ 
\State \textbf{return} $C^{\text{grid}}$, $C^{\text{gridCoords}}$
\end{algorithmic}
\end{algorithm}


\subsection{Compton Scattering}

The Compton scattering algorithm is responsible for simulating the physical process for the event of Compton scattering. Based on the Klein-Nishina formula (Equation~\ref{eq:kleinNishina}) and the rejection sampling method from Section~\ref{sec:physicsComptonScattering}, the algorithm samples the scatter angle and computes the new photon energy and direction after the scattering event.

The algorithm initializes with the following parameters:

\begin{itemize}
    \item The initial energy of the photon $E_i$.
    \item The (unit) direction of the photon $\vec{\omega}_i$.
    \item Two random variables $u_1, u_2 \sim \mathcal{U}(0, 1)$ to sample the
    new photon energy and direction.
\end{itemize}

The algorithm applies the Klein-Nishina procedure to compute the scatter angle
and therefore utilizes the electron rest energy $E_\text{rest}$ and follow this
procedure:

\begin{enumerate}[label=\arabic*.]
    \item Initialize 
    \begin{equation*}
        k = E_i/ mc^2, \qquad \zeta(k) = \frac{2(2k^2+2k+1)}{(2k + 1)^3}, \qquad b(k) = \frac{1+\frac{\zeta}{2}}{1-\frac{\zeta}{2}}, \qquad a(k) = 2(b-1).
    \end{equation*}
    \item Rejection Sampling loop:
        \begin{enumerate}[label*=\arabic*.]
            \item Generate random number $r\sim\mathcal{U}(0,1)$.
            \item Calculate 
            \begin{align*}
                x&=b-(b+1)\left(\frac{\zeta}{2}\right)^r \\
                f(k,x) &= \frac{1+x^2+\frac{k^2(1-x)^2}{1+k(1-x)}}{(1+k(1-x))^2} \\
                h(k,x) &= \frac{a}{b-x} \\
            \end{align*}
            \item \textbf{If} $u_1 \leq \frac{f(k,x)}{h(k,x)}$ \textbf{then}: $\theta = \arccos(x)$
        \end{enumerate}
    \item Calculate photon energy with Equation~\ref{eq:compton_energy} and direction using the azimuthal angle $\phi$ from Equation~\ref{eq:azimuthal_angle}.
\end{enumerate}

Given the scatter angle $\theta$, the new photon energy $E_{i+1}$ is computed as in \cite{nelsoncompton}:
\begin{equation}
    E_{i+1} = \frac{E_i}{1 + k (1 - \cos \theta)}.
\end{equation}

With the random variable $u_2$ an azimuthal angle $\phi$ is sampled to determine the new direction.
\begin{equation}
    \phi = 2\pi u_2
\end{equation}
Assuming the vector $\vec{u}$ is an orthogonal unit vector $\vec{u} \perp \vec{\omega}_i$ and $\vec{v} = \vec{u} \times \vec{\omega}_i$, then the new direction $\vec{\omega}_{i+1}$ is as follows:
\begin{equation}
    \vec{\omega}_{i+1} = \sin(\theta)cos(\phi)\cdot u + \sin(\theta)sin(\phi)\cdot v + \cos(\theta)\cdot \vec{\omega_i}
\end{equation}

This leads to the following return values of the algorithm:
\begin{itemize}
    \item The new photon energy $E_{i+1}$ after the Compton scattering event.
    \item The new (unit) direction $\vec{\omega}_{i+1}$ of the photon after the
    Compton scattering event.
\end{itemize}

In the pseudoalgorithm is more explicitly describing this process in detail in
Algorithm~\ref{alg:comptonScatteringBlank} by implementing omitted tweaks and
details.

\begin{algorithm}[H]
\caption{Compton Scattering Algorithm}
\label{alg:comptonScatteringBlank}
\begin{algorithmic}[1]
\Require Initial photon energy $E_i$, direction $\vec{\omega}_i$
\Require Electron rest energy $E_\text{rest}$
\Require Random variables $u_1, u_2 \sim \mathcal{U}(0, 1)$
\Ensure New photon energy $E_{i+1}$, direction $\vec{\omega}_{i+1}$

\LineComment{Scatter angle sampling}
\State $k \gets E_i / E_\text{rest}$
\State $\varepsilon_0 \gets 1 / (2k + 1)$
\Repeat
    \State Generate $r \sim \mathcal{U}(0, 1)$
    \If{$r < 0.5$}
        \State $\varepsilon \gets \varepsilon_0 + (1 - \varepsilon_0) \cdot 2r$
    \Else
        \State $\varepsilon \gets \varepsilon_0 + (1 - \varepsilon_0) \cdot 2(1 - r)$
    \EndIf
    \State $\cos\theta \gets 1 + \frac{1}{k} \left(1 - \frac{1}{\varepsilon}\right)$
    \If{$|\cos\theta| \leq 1$}
        \State $\sin^2\theta \gets 1 - \cos^2\theta$
        \If{$u_1 \leq \frac{1}{2}\left(\varepsilon + \frac{1}{\varepsilon} - \sin^2\theta\right)$}
            \State $\theta \gets \arccos(\cos\theta)$
            \State \textbf{break}
        \EndIf
    \EndIf
\Until{accepted}

\LineComment{New photon energy calculation}
\State $E_{i+1} \gets \frac{E_i}{1 + k (1 - \cos\theta)}$

\LineComment{Orthonormal basis construction}
\State $\phi \gets 2\pi u_2$
\If{$|(\vec{\omega}_i)_z| < 0.999$}
    \State $\vec{a} \gets (0, 0, 1)$
\Else
    \State $\vec{a} \gets (1, 0, 0)$
\EndIf
\State $\vec{u} \gets \frac{\vec{a} \times \vec{\omega}_i}{\|\vec{a} \times \vec{\omega}_i\|}$
\State $\vec{v} \gets \vec{\omega}_i \times \vec{u}$
\LineComment{New direction calculation}
\State $\vec{\omega}_{i+1} \gets \sin\theta \cos\phi \cdot \vec{u} + \sin\theta \sin\phi \cdot \vec{v} + \cos\theta \cdot \vec{\omega}_i$

\State \Return $E_{i+1}, \vec{\omega}_{i+1}$
\end{algorithmic}
\end{algorithm}


% ------------------------------------------------------------------------------
\section{Algorithm Composition}
\label{sec:algorithmComposition}
% ------------------------------------------------------------------------------
The main algorithm is composing the sub-algorithms from Section~\ref{sec:subAlgorithms}.

The algorithm is initialized with the following parameters:
\begin{itemize}
    \item The voxel grid $materialGrid$ representing the geometry of the scene.
    \item The voxel grid $totalAttenuationGrid$ representing the total
    attenuation coefficients of the materials in the voxel grid $materialGrid$.
    \item The voxel grid $comptonAttenuationGrid$ representing the Compton
    scattering coefficients of the materials in the voxel grid $materialGrid$.
    \item The voxel grid $absorptionGrid$ representing the absorption
    coefficients of the materials in the voxel grid $materialGrid$.
    \item The source position $S$ of the X-ray tube.
    \item The number of scatter events $N$ to simulate.
    \item The opening angle $\alpha$ of the X-ray beam.
    \item The voxel size $s$ of the voxel grid.
    \item A sequence of random variables $u \sim \mathcal{U}(0, 1)^{3(N+1)}$ to
    sample the photon energies, directions and free path lengths.
\end{itemize}

Hereby, the photon generation algorithm is called to generate the photon energies and directions.

Then the photon is initialized with the an initial energy $E_0$, an initial
direction $\vec{\omega}_0$ and an initial intensity $W_0=1$ by applying
Algorithm~\ref{alg:photonEnergySampling} and
Algorithm~\ref{alg:uniformDirectionSampling}. Based on the initial position $A_0
= S$, direction $\vec{\omega}$, the voxel grid and voxel size $s$, the ray
traversal algorithm (Algorithm~\ref{alg:rayTraversal}) is applied to traverse
the photon through the voxel grid until it reaches the first material which is
not air. The exit point of the phantom is denoted as $C_0$.

\subsubsection*{The loop over the scatter events:}

Later $N$ iterations of the forced detection algorithm
(Algorithm~\ref{alg:forcedDetection}) are applied together with one random
variable to simulate the photon transport through the phantom. In each
iteration, the photon position and intensity are updated. With
Algorithm~\ref{alg:exitPointComputation}, the exit point $C_i^\text{grid}$ of
the photon is computed and captured together with the relevant intensity $E_i
\cdot W^{\text{escape}}_{i+1}$.

Then the Compton scatter event is simulated, taking into account the photon
energy $E_i$, the direction $\vec{\omega}_i$ and the Compton scattering
coefficient $\mu_{\text{CS}}$ at $A_{i+1}$. Together with two random variables,
the new photon energy $E_{i+1}$ with an according direction $\vec{\omega}_{i+1}$
is sampled.

\begin{algorithm}[H]
\caption{Main Simulation Algorithm Composition}
\label{alg:mainSimulation}
\begin{algorithmic}[1]
\Require Material grid $materialGrid$
\Require Total attenuation grid $totalAttenuationGrid$
\Require Compton attenuation grid $comptonAttenuationGrid$
\Require Absorption grid $absorptionGrid$
\Require Source position $S$
\Require Number of scatter events $N$
\Require Beam opening angle $\alpha$
\Require Voxel size $s$
\Require Random variables $u \sim \mathcal{U}(0, 1)^{3(N+1)}$
\Ensure Detector contributions (primary and scatter signals)

\LineComment{Photon generation}
\State $E_0 \gets$ Algorithm~\ref{alg:photonEnergySampling}(spectrum, $u_1$)
\State $\vec{\omega}_0 \gets$ Algorithm~\ref{alg:uniformDirectionSampling}($\alpha$, beam axis, $u_2$, $u_3$)
\State $W_0 \gets 1$

\LineComment{Traverse photon through air to phantom}
\State $A_0, \ldots \gets$ Algorithm~\ref{alg:rayTraversal}(throughAir=true, $materialGrid$, $s$, $S$, $\vec{\omega}_0$)

\For{$i = 0$ to $N-1$}
    \LineComment{Forced detection step}
    \State $A_{i+1}, W_{i+1}, W^\text{escape}_{i}, C^\text{phantom}_{i}, \mu_\text{CS} \gets$ Algorithm~\ref{alg:forcedDetection}($materialGrid$, $totalAttenuationGrid$, $comptonAttenuationGrid$, $absorptionGrid$, $A_i$, $\vec{\omega}_i$, $E_i$, $W_i$, $u_{3i+1}$, $s$)
    
    \LineComment{Record escaped photon contribution}
    \If{$W^\text{escape}_i > 0$}
        \State $C^\text{world}_i, \ldots \gets$ Algorithm~\ref{alg:exitPointComputation}($C^\text{phantom}_i$, $\vec{\omega}_i$, grid shape, $s$)
        \State Add $E \cdot W^\text{escape}_i$ to detector pixel at $C^\text{world}_i$
    \EndIf

    \LineComment{Compton scatter step}
    \State $E_{i+1}, \vec{\omega}_{i+1} \gets$ Algorithm~\ref{alg:comptonScatteringBlank}($E$, $\vec{\omega}_i$, $\mu_\text{CS}$, $u_{3i+2}$, $u_{3i+3}$)
\EndFor

\LineComment{Handle leftover intensity}
\State $C^\text{world}_N, \ldots \gets$ Algorithm~\ref{alg:exitPointComputation}($A_N$, $\vec{\omega}_N$, grid shape, $s$)
\State Add $E \cdot W$ to detector pixel intensity at $C^\text{world}_N$

\end{algorithmic}
\end{algorithm}
